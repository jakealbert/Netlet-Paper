\subsubsection{Ease of Use}
Poor design is a critical barrier to the adoption of technologies. A product can offer tremendous power and an exceptional feature set but fail because its features are inaccessible or incomprehensible to the user. Conversely, good design can, and has frequently been, the determining factor in the ascendancy of a product---and this occurs far more than many members of engineering academia acknowledge. To take one widely cited example, Apple's iPod was neither more feature-rich nor cheaper than other mp3 players available at the time it debuted on the market. Its ease of use and (admittedly well-marketed) elegant design were its only distinguishing attributes.\\

For this reason, we committed to implementing the \netlet's features in the simplest and most accessible way possible. Our target use case differed from that of an ordinary non-smart power strip (i.e. buy a power strip, take it home, switch it on, and plug appliances into it) in a single way: namely, the user's ability to follow the plugging in of appliances by logging onto a website, registering the serial number of their \netlet, and controlling it from that web interface. \netlet's current and power monitoring, its remote switching, and its spike detection would all be seamlessly available to the user with the barest minimum of configuration. Moreover, all these networked features would be secured with industry-standard encryption out of the box.\\

We encountered only one major obstacle to this goal of seamless connectivity. Many if not most home \wifi networks are secured with some kind of encryption requiring a password. Furthermore many \wifi networks may be available in a single place. Specifying the name and password of the target wireless network is necessary to enable the \netlet's connectivity. Due to constraints on price and complexity it proved infeasible to prototype this connectivity (although a  method of transferring authentication details via a usb key was considered). As an alternative, the prototype \netlet uses \ethernet to achieve its connectivity. While this is not an ideal solution with respect to our target use case, we deemed it plausible enough for the prototype.\\
