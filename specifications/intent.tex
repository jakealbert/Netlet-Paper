\subsection{Intent}

Netlets are the fruit of a simple realization: there is much to be done in the area of power conservation. While concerns grows with every dollar added to the barrel of crude oil, the four of us considered ways in which embedded systems could contribute to lowering our carbon footprints.\\

Much of the energy wasted in a house is squandered in three ways: by leaving things on that should be turned off (e.g. lights, printer) , by using inefficient appliances (e.g. incandescent light-bulbs) and by using appliances inefficiently (heating a poorly insulated house, running the dishwasher when it is half full). While some of these issues do not lend themselves well to automation, they could be remedied by educating users and homeowners of the cost of those inefficient and inefficiently used appliances.\\

Due to their position in the distribution systems, power strips are the logical place to start. By modifying them, we can monitor and control power consumption without having to modify each and every appliance we own. Furthermore, their low cost makes it realistic for a user to start using netlets around his house and achieve a quick return on investment.\\

Netlets therefore are more compact fluorescent light-bulbs than they are Toyota Prius. Capitalizing on the decreasing price of computing power, we hope to market them not only to the environmentally conscious upper class, but also to the financially motivated middle class.\\

Netlets is also an interface. Beyond the power strip, we hoped to create a simple protocol for appliances to use to report data and receive commands. \textit{to be expanded.}\\
