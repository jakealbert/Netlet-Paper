\subsubsection{Main Features}

\paragraph{Monitoring Current} \netlet's first main feature is its capacity to monitor the current passing through it in real time. By sampling the output of a current sensor through which that current passes, the \atmega is able to digitize a discrete-time representation of the current as it changes through time. The RMS average of this signal is taken as an indication of the amount of current being drawn by the appliance attached to the \netlet.

\paragraph{Calculate Power Use} \netlet's second main feature is its ability to compute power consumption statistics. Because the standard voltage for mains power in America is 120VAC, it is safe to assume this is the actual voltage of the power line switched and monitored by the \netlet, although this is not an ideal solution. Because of time and resource constraints, though, we use this assumption to compute the real-time power consumption based on the RMS current passing through the \netlet. Although our initial plan was to use an isolated and attenuated signal taken directly from mains power to measure the exact voltage curve and to use this in conjunction with the current curve to compute the phase difference between the signals as well as the precise real and apparent power consumption of the device being switched. The relatively technical (and not consumer-friendly) nature of this statistic as well as the complexity of isolating and measuring a voltage curve both weighed heavily in our decision to approximate power consumption by only measuring current.

\paragraph{Switch On/Off} \netlet's third main feature is its ability to switch power to any of its outlets in response to a signal from the server. This feature is perhaps the most consumer-friendly of them all: users would be able to control their lights from their handheld computers or cellular telephones while sitting in bed, or to switch off appliances they had mistakenly left on without returning home. The server can also be instructed to automatically switch outlets at given times or to switch them in response to power consumption events. For example, the server could switch a computer's peripherals off when the computer itself stops drawing power.

\paragraph{Spike Detection} \netlet's last main feature is its constant monitoring for spikes in power consumption. Rising or falling edges in the current indicate the device was switched on/off or entered/exited ``sleep'' mode.  My relaying spike information back to the controller, \netlets allows non-host devices to smartly be disconnected from the power grid when the host  device is not in use.  For example, printers and scanners can be unplugged when the computer hibernates, and toasters, food processors, and coffee makers can be unplugged when the kitchen lights are off.




